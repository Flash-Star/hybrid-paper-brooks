%\documentclass[12pt,preprint]{aastex}
\documentclass[iop,apj]{emulateapj}
%\usepackage{multirow}
\usepackage{longtable}
\usepackage{ulem}
%\usepackage[monochrome]{color}
\usepackage{color}
%\usepackage{lipsum}
\usepackage{amsmath}
%\usepackage{hyperref}

\newcommand{\eqqref}[1]{Equation (\ref{#1})}
\newcommand{\tabref}[1]{Table~\ref{#1}}
\newcommand{\figref}[1]{Figure~\ref{#1}}
\newcommand{\secref}[1]{Section~\ref{#1}}
\newcommand{\appref}[1]{Appendix~\ref{#1}}

\newcommand{\SNeIa}{SNe~Ia}
\newcommand{\SNIa}{SN~Ia}
\newcommand{\C}[1]{\ensuremath{{}^{#1}{\rm C}}}
\newcommand{\Ox}[1]{\ensuremath{{}^{#1}{\rm O}}}
\newcommand{\Ne}[1]{\ensuremath{{}^{#1}{\rm Ne}}}
\newcommand{\Na}[1]{\ensuremath{{}^{#1}{\rm Na}}}
\newcommand{\Mg}[1]{\ensuremath{{}^{#1}{\rm Mg}}}
\newcommand{\Ni}[1]{\ensuremath{{}^{#1}{\rm Ni}}}
\newcommand{\Co}[1]{\ensuremath{{}^{#1}{\rm Co}}}
\newcommand{\Si}[1]{\ensuremath{{}^{#1}{\rm Si}}}
\newcommand{\Fe}[1]{\ensuremath{{}^{#1}{\rm Fe}}}
\newcommand{\code}[1]{\textsc{#1}}
\newcommand{\FLASH}{\code{FLASH}}
\newcommand{\CASTRO}{\code{CASTRO}}
\newcommand{\MESA}{\code{MESA}}
\newcommand{\PARAMESH}{\code{PARAMESH}}
\newcommand{\pv}{\ensuremath{\phi}}
\newcommand{\bvec}[1]{\ensuremath{\boldsymbol{#1}}} %boldface vector style
\newcommand{\grad}{\bvec{\nabla}} %gradient
\newcommand{\curl}{\bvec{\nabla \times}} %curl
\newcommand{\Atwood}{\ensuremath{\mathrm{At}}}
\newcommand{\adndt}{At.~Data~Nucl.~Data~Tables}
\newcommand{\At}{{\rm At}}
\newcommand{\ee}[1]{\ensuremath{\times 10^{#1}}}
\newcommand{\cdens}{\rho_{c}}

% basic unit typesetteing
\newcommand{\unitspace}{\ensuremath{\,}}
\newcommand{\usp}{\unitspace}
\newcommand{\numberspace}{\ensuremath{\;}}
\newcommand{\nsp}{\numberspace}
\newcommand{\unitstyle}[1]{\ensuremath{\mathrm{#1}}}
\newcommand{\power}[2]{\ensuremath{{#1}^{#2}}}


% prefixes
\newcommand{\nano}{\unitstyle{n}}
\newcommand{\milli}{\unitstyle{m}}
\newcommand{\centi}{\unitstyle{c}}
\newcommand{\kilo}{\unitstyle{k}}
\newcommand{\Mega}{\unitstyle{M}}
\newcommand{\Giga}{\unitstyle{G}}

% base units, mks
\newcommand{\meter}{\unitstyle{m}}
\newcommand{\kilogram}{\kilo\gram}
\newcommand{\second}{\unitstyle{s}}

\newcommand{\Kelvin}{\unitstyle{K}}
\newcommand{\K}{\Kelvin}  %degrees Kelvin


% base units, cgs
\newcommand{\cm}{\centi\meter}
\newcommand{\gram}{\unitstyle{g}}


% derived units
\newcommand{\grampercc}{\gram\usp\power{\cm}{-3}} %mass density
\newcommand{\grampersquarecm}{\gram\usp\power{\cm}{-2}} %column depth
\newcommand{\GramPerCc}{\grampercc}
\newcommand{\GramPerSc}{\grampersquarecm}
\newcommand{\columnunit}{\grampersquarecm}
\newcommand{\dyne}{\unitstyle{dyn}} %dyne
\newcommand{\erg}{\unitstyle{ergs}} %ergs
\newcommand{\ergs}{\erg}
\newcommand{\gauss}{\unitstyle{G}} %gauss
\newcommand{\ergspersecond}{\erg\unitspace\power{\second}{-1}}
\newcommand{\ergspergram}{\erg\unitspace\power{\gram}{-1}}
\newcommand{\cgsflux}{\erg\unitspace\power{\cm}{-2}\usp\power{\second}{-1}}
\newcommand{\kms}{\kilo\meter\unitspace\power{\second}{-1}}

% Nuclear and atomic units
\newcommand{\amu}{\unitstyle{u}} %atomic mass unit
\newcommand{\angstrom}{\mbox{\AA}} %Angstrom
\newcommand{\fermi}{\unitstyle{fm}} %fermi
\newcommand{\eV}{\unitstyle{eV}}        %eV
\newcommand{\keV}{\kilo\eV} %Kev
\newcommand{\MeV}{\Mega\eV} %MeV

% solar and astronomical units
\newcommand{\Msun}{\ensuremath{M_\odot}}
\newcommand{\Myr}{\Mega\yr}
\newcommand{\Gyr}{\Giga\yr}
\newcommand{\parsec}{\unitstyle{pc}}
\newcommand{\kpc}{\kilo\parsec} %kiloparsec
\newcommand{\mJy}{\unitstyle{\mu Jy}} %micro Jansky

% misc. units
\newcommand{\minute}{\unitstyle{min}} %minute
\newcommand{\hour}{\unitstyle{hr}} %hour
\newcommand{\yr}{\unitstyle{yr}}        %year
\newcommand{\km}{\kilo\meter}   %kilometers
\newcommand{\Hz}{\unitstyle{Hz}}        %Hertz
\newcommand{\ksec}{\kilo\second} %kilosecond

\newcommand{\tDDT}{\ensuremath{t_{\rm DDT}}}
\newcommand{\rhoDDT}{\ensuremath{\rho_{\rm DDT}}}
\newcommand{\COreac}{\ensuremath{\C{12}\left(\alpha,\gamma\right)\Ox{16}}}

\bibliographystyle{apj}

\shorttitle{Hybrid Ia Progenitors}

\begin{document}

\title{Type Ia Supernova Explosions from Hybrid Carbon-Oxygen-Neon White Dwarf Progenitors}

\author{
Carlyn N.\ Augustine\altaffilmark{1},
Donald E.\ Willcox\altaffilmark{2},
Dean M.\ Townsley\altaffilmark{3},
and Alan C.\ Calder\altaffilmark{2,4}
}

\altaffiltext{1}{
  Department of Physics and Astronomy,
  The University of Alabama, Tuscaloosa, AL, 35487-0324, USA
}
\altaffiltext{2}{
  Department of Physics and Astronomy,
  Stony Brook University, Stony Brook, NY, 11794-3800, USA; \\
  \href{mailto:donald.willcox@stonybrook.edu}{donald.willcox@stonybrook.edu}
}
\altaffiltext{3}{
  Institute for Advanced Computational Sciences,
  Stony Brook University, Stony Brook, NY, 11794-5250, USA
}

\begin{abstract}
Recent studies of stellar evolution predict the existence of "hybrid" white 
dwarfs, made of a C/O/Ne
core within a O/Ne shell, and that these are viable progenitors for supernovae.
More recent work found that the C/O core is mixed with the surrounding O/Ne
while the white dwarf cools. Inspired by this scenario, we performed simulations of
thermonuclear supernovae from these hybrid progenitors. The progenitor model 
was constructed with the one-dimensional stellar evolution code MESA and
represented a star evolved through the phase of unstable interior mixing 
followed by accretion until it reached conditions for the ignition of 
carbon burning. The MESA model was then mapped to a two-dimensional initial 
condition and an explosion simulated with FLASH. For comparison, a similar
simulation of an explosion was performed from a traditional C/O progenitor
white dwarf. By comparing the yields of the explosions, we found....
how explosions from these models differ from explosions from previous models
without the mixing during the white dwarf cooling.
\end{abstract}

\keywords{hydrodynamics --- nuclear reactions, nucleosynthesis, abundances
--- supernovae: general --- white dwarfs}

%%%%%%%%%%%%%%%%%%%%%%%%%%%%%%%%%%%%%%%%%%%%%%%%%%%%%%%%%%%%%%%%%%
\section{Introduction}
\label{sec:intro}
Thermonuclear (Type Ia) supernovae (\SNeIa) are bright stellar explosions 
thought to occur when approximately 1.X \Msun\ of material composed principally 
of C and O burns under degenerate conditions (SETTLE ON X). This class of supernovae is
know to synthesize much of the Fe-group elements found in the galaxy and
the light curves of these events have a spectial property that allows
their use as distance indicators for cosmological studies~\citep{phillips:absolute}
resulting in the discovery of the acceleration of the expansion of
the Universe and thus the inference of Dark 
Energy~\citep{riess.filippenko.ea:observational,
perlmutter.aldering.ea:measurements,leibundgut2001}. 
This special property of the light curve is thought to follow
from the fact that the source of luminosity, the radioactive decay
of \Ni{56}, synthesized by the thermonuclear burning, is also the
principle source of opacity~\citep{Pinto2001The-type-Ia-sup}. 

Supernovae are classified observationally 
by their light curves and spectra~\citep{minkowski41,bertola64,porterfilippenko87,
wheelerharkness1990conf,Fili97}, with the type Ia designation following from
the absence of H in the spectrum and the presence of a specific Si 
line~\citep{filippenko:optical,hillebrandt.niemeyer:type}. These events
have been associated with C burning under degenerate conditions
for some time~\citep{hoylefowler60,arnett.truran.ea:nucleosynthesis},
but discerning the setting(s) of these events is proving difficult
and remains the subject of active research. At present there are three
widely-accpeted scenarios: the {\em single degenerate} scenario,
the {\em double detonation} or {\em sub-Chandrasekhar} scenario, 
the {\em white dwarf merger} or {\em double degenerate} scenario.
We briefly describe these in the subsection that follows. `
Also see~\citet{hillebrandt.niemeyer:type,howell2011,hillebrandtetal2013,calderetal2013,roepkesim2018}
for additional discussion.

\subsection{Proposed Progenitor Settings}

The single degenerate picture posits a white dwarf gaining mass
from a main-sequence companion. The process relies on a long
period of accretion combined with either steady burning or a 
series of nova explosions to allow the WD to gain
$\sim 0.6$ CHECK THIS \Msun\ needed for it to approach the
limiting Chandrasekhar mass~\citep{starrfieldetal2012}. As it approaches
the Chandrasekhar limit, conditions in the compressed core are right
to ignite the thermonuclear buring that will incinerate the star. 

Within this progenitor setting, many mechanisms for the explosion have 
been studied with the result being that the outcome is sensitive to
the ignition conditions and the nature of the thermonuclear buring.
Studies indicate that neither a pure deflagration (subsonic flame front)
nor a pure detonation (supersonic burning front) produce ejecta conisitent
with observations \citep{arnett69,roepkeetal07}.
Instead, the models that best reproduce the observed
stratified ejecta are those in which the burning begins as a 
subsonic deflagration, which allows the star to react and expand
thus lowering the density, that is followed by a 
detonation~\citep{Nomo84,Khokhlov1991Delayed-detonat,HoefKhok96,GameKhokOran05}.
The ``classic'' delayed detonation model is that of 
Khokhlov \cite{Khokhlov1991Delayed-detonat,hoflich.khokhlov.ea:delayed,GameKhokOran05},
and many variations have been 
explored \cite[and references therein]{hillebrandtetal2013,calderetal2013}.
The approach we employ for this study is a variation, the 
deflagration-to-detonation transition (DDT), in which the transition
occurs at the top of a rising Rayleigh-Taylor unstable plume of hot, burned 
material when it reaches a threshold density.
Which we describe this approach and our methodology  in detail below.

The double detionation picture is a variation of the single degenerate picture.
In this case, a white dwarf accretes material from a companion, but rather
than gaining enough mass to approach the Chandrasekhar limit, a detonation
occurs in the accreted layer that subsequently triggers a detonation
in the underlying white dwarf \citet{woosleyweavertaam80,taam80a,taam80b,
nomoto80,nomoto82b}. Early studies indicated that a detonation in an 
accreted He
layer could produce an inwardly propagating
shock that would ignite a detonation in the C-O core and found
that this scenario could work for a wide range of white dwarf
masses, not just the near-Chandrasekhar case \citep{livne90}, hence the
moniker ``sub-Chandrasekhar'' \citep{ww94}. Multidimensional supported the
efficacy of this picture~\citep{livneglasner91, livnearnett95,HoefKhok96,
hoeflichetal96, wigginsfalle97,wigginsetal98,garciasenzbravowoosley99} 
but indicated that uncertainties 
like the initial conditions were found to play a significant role.
on the explosion outcome. 

A particular problem was that most models 
included a massive He layer, which leads to 
the unobserved result intermediate and heavy elements synthesized in 
the He detonation in the outer parts of the ejecta~\citep{HoefKhok96, 
hoeflichetal96,finkhillebrandtroepke2007,simetal2010}.
\citet{bildstenetal2007}, however, found that fairly thin He layers
could flash on sub-Chandrasekhar
mass white dwarfs, partially resolving the issue and encouraging
further research~\citep{simetal2012,brooksetal2015, shenetal2018,
glasneretal2018}.

The white dwarf merger progentor picture has two white dwarfs coming 
together and subsequently exploding~\citep{tutukovyungelson76,tutukovyungelson79,
webbink84,ibentutukov84}. This scenario provides an abundance of degenerate fuel, 
which may explain some bright events~\citep{scalzo:2010,Yuan:2010}. 
An early concern about this model came from studies finding that
the white dwarf will ignite at the edge and the coalesced object, 
and a flame will burn inward, converting the C-O white dwarf into 
an O-Ne-Mg white dwarf~\citep{saionomoto1985,saionomoto2004}, with
further accretion leading to the collapse of the white dwarf
into a neutron star, a scenario termed ``accretion-induced collapse''
\citep{nomotokondo1991}.

Subsequent work allayed this concern~\citep{yoonetal2007,lorenaguilaretal2009,
Shenetal12, pakmoretal2012b} and motivated by the need to expain briht events
and the estimated population WD pairs, active research continues. Contemporary
research focuses on variations on the merger idea, including 
inspiraling pairs, collisions, violent mergers,
and the ``core-degenerate'' model in which the merger takes place in
a common envelope \citep{raskinetal2009,pakmoretal2011,kashi:2011,pakmoretal2012a,Shenetal12,katzetal2016}.


\subsection{The Deflagraiton to Detonation Transition Mechanism Within the Single Degenerate Scenario}

Our models are within the DDT explosion paradigm \cite{1986SvAL,
Khokhlov1991Delayed-detonat,NiemWoos97,Niem99,belletal2004,fishjump2015}. 
In this scenario, the accretion of mass on the white dwarf compresses and 
heats the core, igniting carbon fusion and driving a period of convection
\citep{WoosWunsKuhl04,wunschwoosley2004,Kuhletal06,nonakaetal2012}.
At some point, the fusion rate becomes fast enough due to the rising
temperature that energy production exceeds convective cooling and
the deflagration phase begins in the core~\citep{Nomo84,WoosWunsKuhl04}.
This flame is unstable, and as the deflagration propagates toward the surface
of the WD, it is subject to the Rayleigh-Taylor instability that generates
turbulence and boosts buring.  
Burning proceeds as a deflagration for about one second, and then 
the flame transitions to a detonation~\citep{hoflich.khokhlov.ea:delayed}. 


%Maybe add some of below?
%and variations include pulsational
%detonations \cite{ivanovaetal1974,Khokhlov1991Delayed-detonat,
%arnettlivne94a, arnettlivne94b, bravogarcia-senz2006},
%gravitationally confined detonation \cite{PlewCaldLamb04,Jordan2008Three-Dimension,
%jordanetal2012}, and deflagration-to-detonation
%transitions (DDTs) \cite{1986SvAL,Khokhlov1991Delayed-detonat,NiemWoos97,Niem99,belletal2004,
%fishjump2015}. We adopt a variation of the latter case, a deflagration-to-detonation transition
%that occurs at the top of a rising plume of hot, burned material when it
%reaches a threshold density.

The mechanism by which the deflagration transistions to a detonation
is imcompletely understood, but the development of fluid instabilities
at the deflagration front, particuarly the Rayleigh-Taylor 
instability~\cite{taylor+50,chandra+81}, is an accepted part of the process.
Many ideas have been proposed, including the deflagration front
entering a regime of distributed burning and the net
rate becoming supersonic \cite{NiemWoos97}, the unstable
flame reaching a certian fractal dimension~\cite{woosley90},
and the Zel'dovich mechanism in which condtions are right for
a pocket of fuel to 
detonate~\cite{zeldovichetal1970,KhokOranWhee97,jacketal2014}.
Our simulations assume that the transition occurs when the top of a 
rising, Rayleigh-Taylor unstable plume reaches a characteristic low
density \cite{townsley.calder.ea:flame}.

%
%
%Explain the Single Degenerate scenario
%\citep{Baraffe2004Stability-of-Su,WoosWunsKuhl04,wunschwoosley2004,Kuhletal06,nonakaetal2012}.
%expansion \citep{Nomo84,WoosWunsKuhl04}, and a flame is born. 
%
%Explain DDT
%
%Go through below and cite some of these as you tell the story
%
%d~\citep{arnett.truran.ea:nucleosynthesis}.
%~\citep{mazzalietal2008}.  
%
%
%\citep{khokhlov1993,bychovliberman1995,
%SNrt,Khok95,NiemHill95,khoketal1997,ZingDurs07,
%cholazarianvishniac2003,
%roepkehn2003,roepkehn2004,
%Zingale2005Three-dimension,Schmetal06a, Schmetal06b,Aspdetal08,
%Woosetal09,csetal2009,hicksrosner2013,c-ssr2013,
%jacketal2014,poludnenko2015,hicks2015}.
%
%
%A deflagration alone will not produce a event of normal brightness and
%expansion velocity~\citep{roepkeetal07}. Instead, the initial
%deflagration must transition to a detonation after the star has
%expanded some in order to produce abundances and a stratified ejecta
%in keeping with
%observations~\citep{Khokhlov1991Delayed-detonat,hokowh95}.
%The physics of this ``deflagration-to-detonation transition'' (DDT)
%are not completely understood, but there has been considerable study
%based on mechanisms involving flame fronts in highly turbulent
%conditions~\citep{1986SvAL, woosley90, Khokhlov1991Delayed-detonat,
%  hokowh95, HoefKhok96, khoketal1997, NiemWoos97, hwt98, Niem99,
%  GameKhokOran05,roepke07, poletal2011,c-ssr2013,poludnenko2015}.
%These models generally reproduce the observations under certain
%assumptions about the ignition~\citep{townetal2009}, but research has
%shown that the results are very sensitive to the details of the
%ignition~\citep{PlewCaldLamb04,GameKhokOran05,garciasenz:2005,
%  roepkeetal07,Jordan2008Three-Dimension}.

\subsection{A Recent Advance in Stellar Evolution: Hybrid White Dwarfs}

Modern computing resources now enable simulations with unprecedented 
realism, allowing both one-dimensional simulations with a vast
amount of included physics and full three-dimensional simulations
albeit with less included physics~\cite{caldertownsley2018}.
In the area of stellar evolution, recent investigations revisiting 
late-time evolution of roughly 8 \Msun stars indicate that under
the right circumstances, ``hybrid" white dwarfs having a C/O core surrounded by O/Ne 
mantle may form \citep{siess2009,denissenkovetal2013}. These hybrid
white dwarf are thought to form when mixing at the lower convective boundary quenches
C burning in an asymptotic giant branch (AGB) star, leaving unburned C
in the core. The situation is at best uncertain, however, and the 
results depending on assumptions about convective
overshoot that have been questioned \citep{chenetal2014,lecoanetetal16,lattanzioetal2017}.

Assuming a hybrid WD forms, there is subsequent evolution that will occur. There
are two ways in which a hybrid WD can become a progenitor of a thermonuclear supernova.
First, the hybrid could be part of a binary system in which the companion star becomes
another white dwarf and the two merge. Recent population synthesis work supports
this pathway by indicating a substantial contribution to the Ia population rate from 
mergers where one member is a hybrid~\citep{yungelsonkuranov2017}. The second way
occurs if the hybrid WD is part of a binary with a main sequence or giant companion
and gains enough mass to approach the Chandrasekhar mass and ignite C fusion (i.e.
the single degenerate picture)~\citep{}. 

In either case, the hybrid WD will experience a period of cooling, which will
make the core-mantle interface subject to convective 
instability \citep{brooksetal2017,schwabgaraud2018}. In the case of the 
hybrid WD accreting and approaching the Chandrasekhar mass, accretion will
heat the core and start C fusion, leading to a period of ``simmering" prior
to the explosion \citep{who?}. The upshot is that is likely to be considerable
mixing after the hybrid forms that may homegenize the compostion 
\citep{denissenkovetal2015,brooksetal2017,schwabgaraud2018}.


Hybrid WDs have more mass than traditional C/O WDs, with some studies indicating the
mass can approach 1.3 \Msun \citep{chenetal2014}. This increased mass
minimizes one of the problems associated with the single-degenerate picture,
the need to accrete enough mass for the WD to approach the Chandrasekhar
mass~\citep{chenetal2014,denissenkovetal2015,kromeretal2015}.
Accordingly, there has been considerable interest in viability of explosions from 
these progenitors. 

From population synthesis, \citet{mengpods2014} found that these
%meng received may 5 accepted may 16
progenitors may substantially contribue to the population of SNeIa (1-8\%) and have
relatively short delay times. They also suggested that these 
may produce part of the Iax class of events. \citet{Wangetal2014} also with population
synthesis studied the case
%received august 28 accepted sept 26
of a hybrid progenitor accreting from a nondegenerate He star and found
birth rates indicating that up to 18\% of SNe Ia may follow from this chanel
and very short delay times. \citet{Wangetal2014} also suggested that explosions 
from hybrid progenitors may provide an explanation for type Iax events.
\citet{mengpods2018}, from the common-envelope-wind model developed in 
\citep{mengpods2014}, propose that both Ia-CSM and Iax events 
are caused by the explosion of hybrid progenitors, with Ia-CSM occuring in systems with 
a massive common envelope and Iax events occuring in systems where most of the common envelope
has been lost.

Other groups have simulated explosions from hybrid progenitors. 
\citet{kromeretal2015} performed pure deflagration simulations from models 
with a C core. They found that their models may explain some faint events 
such as SN 2008ha~\citep{foleyetal2009}.
\citet{bravoetal2016} performed one-dimensional simulations of explosions from 
a variety of progenitor models assuming both pure deflagration and the DDT 
explosion mechanism. 
Some of their models are similar to those of \citep{denissenkovetal2015} and 
they report that many models produce less synthesized
\Ni{56}, indicating dimmer events. They also note that some of their
models may explain Iax events.
\citet{willcoxetal2016} simulated explosions from the progenitors of 
\citep{denissenkovetal2015} and found a trend of lower energy and lower
\Ni{56} yield when compared to explosions from traditional C/O progenitors.


If we refer to the work by name, it should appear ``The hybrid models
of~\citet{brooksetal2017}." Or, we can just cite it~\citep{brooksetal2017}.




\section{Methods}

Summarize what we do a la:
In our simulations, we initialize a detonation once the 
deflagration front reaches a
characteristic DDT fuel density, which controls the degree of
expansion the star undergoes during the deflagration stage.

The comparison study of tradition C/O progenitor WDs used the same initial
conditions for the realizations within the suite. The traditional models
are simulated with the same code and the same set of perturbations on the
initial match head (realization). The only difference is the different
nuclear burning to include the other Ne and the initial profile of the
1-d model. 

The comparison study of tradition C/O progentor WDs used the same
initial conditions for the realizations within the suite. Make the
point that the traditional models are simalted with the same code
and the same set if perturbations on the initial match head (that is
what we mean by a realization). The only difference is the different
nuclear buring to include the other Ne and the initial profile of
the 1-d model.  

For much of the methodology to be described below, just say a few
words about what was done and cite \citet{willcoxetal2016}.

\subsection{Initial Conditions}

A hybrid white dwarf model constructed with the one-dimensional stellar 
evolution code MESA~\citep{mesa1,mesa2,mesa3,mesa3e} served as the 
initial conditions for the simulations of supernova explosions presented
in this work~\citep{brooksetal2017}.
The hybrid model was constructed to have conditions as similar a possible
to a ``classic" C/O model and the two shared the same central temperature
and density.  Figure~\ref{fig:init_conds} shows the density and temperature
profiles of the two initial one-dimensional models. 
\begin{figure}
\includegraphics[width=\columnwidth]{figures/initial_conds.png}
\caption{\label{fig:init_conds}
	We used MESA, a 1-D stellar evolution code, to create the hybrid model. We made the hybrid model have the same central temperature and density as the classic model. 
 We found that the most abundant isotopes in the model were C12, O16, Ne20. There were a few other isotopes that were added to form Ne22 later. We saw that  ….
Plotting the density verses solar radii, we found that the densest part of the star is in the center slowly decreasing until it hits 0 at the outer edge of the star. 
In order to see where we should ignite the flame, we plotted the temperature profile. It was found that the peak temperature is at the center of the star, therefore, it was obvious to ignite the flame in the core. (PUT IN TEMP PROFILE)
}
\end{figure}
In both models, and unlike the hybrid models of~\citet{willcoxetal2016}, the central
temperature was the highest at the center of the star and accordingly the simulations 
began with central initions. 


\subsection{Created two-dimensional models from one-dimensional models}

The two-dimensional simulations were performed with a modified version
of the FLASH code developed at the University of Chicago. CITE
The initial one-dimensional MESA models were mapped to the two-dimensional 
FLASH grid. MORE HERE.

An additional step in constructing the two-dimensional initial conditions
was aggregating nuclides representing neutron-rich metals. 
The most abundant isotopes in the model were C12, O16,
Ne20 CHECK NE@) IS SYMMETRIC, which are symmetric (the number of neutrons 
equals the number of protons). Similarly to~\citet{willcoxetal2016}, other 
neutron-rich isotopes in
the initial model were to form Ne22, which serves as a proxy for metallicty. 
MESA model into FLASH, which is a 2-D code, we needed symmetric nuclear
matter. This means that the ratio of neutrons to protons in the nuclei
must be the same ratio as in the mesa profile. This is because the amount
of extra neutrons around is important when burning the star because that
determines how much neutron rich isotopes the burning produces.

\subsection{DDT Process and Suits of Explosions}

The fuel starts off cold. The elections in the plasma conduct heat and the
fuel begins to heat up. Once the fuel reaches a certain temperature, the
reaction starts. The fuel abundance begins to decline because it is consuming
fuel. Energy is produced which makes the fuel hot. Eventually, all that’s
left is ash. As surface area grows, conducted energy increases and density
decreases. At low densities Rayleigh-Taylor instability causes the flame to
get more tangled and unstable. Eventually, the star will get a large enough
surface area in this volume that the net burning effect is supersonic.
Therefore, the transition from deflagration to detonation is made. 

Using the FLASH code from University of Chicago, simulations of
explosions were preformed. For each simulation we varied the central
ignition seed. 

----updated text-----

DDT Process:
The fuel starts off cold. The elections in the plasma conduct heat
and the fuel begins to heat up. Once the fuel reaches a certain
temperature, the reaction starts. The fuel abundance begins to
decline because it is consuming fuel. Energy is produced which
makes the fuel hot. Eventually, all that’s left is ash. As
surface area grows, conducted energy increases and density
decreases. At low densities Rayleigh-Taylor instability causes
the flame to get more tangled and unstable. Eventually, the
star will get a large enough surface area in this volume that
the net burning effect is supersonic. Therefore, the transition
from deflagration to detonation is made.

A deflagration alone will not produce an event of normal
brightness and expansion velocity. Instead, the initial
deflagration must transition to a detonation after the star has
expanded some in order to produce abundances and a stratified
ejecta in keeping with observations. The physics of this
“deflagration-to- detonation transition” (DDT) are not
completely understood, but there has been considerable study
based on mechanisms involving flame fronts in highly turbulent
conditions. These models generally reproduce the observations
under certain assumptions about the ignition, but research has
shown that the results are very sensitive to the details of the
ignition. In our simulations, we initialize a detonation once
the deflagration front reaches a characteristic DDT fuel density,
which controls the degree of expansion the star undergoes during
the deflagration stage.

The comparison study of tradition C/O progenitor WDs used the same initial
conditions for the realizations within the suite. The traditional models
are simulated with the same code and the same set of perturbations on the
initial match head (realization). The only difference is the different
nuclear burning to include the other Ne and the initial profile of the
1-d model.

----above updated text-----



\section{Results}

A plot of the estimated \Ni{56} yield as a function of the final, burned mass
for all CO and Hybrid runs. The Hybrid model is shown in red, and the CO
model is shown in blue. As it is shown, the CO model has a higher final mass
then the Hybrid model. 

The cumulative distribution of the \Ni{56} yield for the C/O and hybrid
simulations is presented in \ref{fig:cumdist}. 
The figure presents the estimated \Ni{56} yield as a function of the final burned 
mass. The results show that the C/O models have a higher burned mass than the
hybrid models for a given estimated mass of \Ni{56}. CHECK THIS.  


\begin{figure}
\includegraphics[width=\columnwidth]{figures/ni56_yield_cum_dist.png}
\caption{\label{fig:cumdist}
A plot of the estimated \Ni{56} yield as a function of the final, burned mass
for all CO and Hybrid runs. The Hybrid models are shown in red, and the CO
models are shown in blue. As it is shown, the CO model has a higher final mass
then the Hybrid model. 
}
\end{figure}

\begin{figure}
\includegraphics[width=\columnwidth]{figures/ni56_vs_time_hybrid.png}
\caption{\label{fig:nithybrid}
Estimated \Ni{56} mass vs.\ time for ten hybrid model explosion simulations. 
Shown are realizations 21-30. 
}
\end{figure}

\begin{figure}
\includegraphics[width=\columnwidth]{figures/ni56_vs_time_CO.png}
\caption{\label{fig:nitco}
Estimated \Ni{56} mass vs.\ time for ten C/O model explosion simulations.
Shown are realizations 21-30. 
}
\end{figure}

The figures show that the final Estimated \Ni{56} for the CO model is
slightly larger then the Hybrid model. Therefore, the CO produces more
\Ni{56}. The Hybrid model has a larger range then the CO model. Also, on
average, the CO model reaches the DDT phase sooner then the Hybrid model.


This is a plot of the estimated \Ni{56} yield vs. mass below 2e7 g/cc at
the time of the DDT.

The degree of expansion was found using the time at which the first
detonation point occurred, and the mass at density less then 2e7 g/cc. The
estimated \Ni{56} Yield is the final estimated mass of the model.

\begin{figure}
\includegraphics[width=\columnwidth]{figures/ni56_yield_vs_mass_at_high_dens_v2.png}
\caption{\label{fig:masshighdens}
This is a plot of the estimated Ni56 yield vs. mass below 2e7 g/cc at the time of the DDT. 
The degree of expansion was found using the time at which the first detonation point occurred, and the mass at density less then 2e7 g/cc. The estimated Ni56 Yield is the final estimated mass of the model. 
Hybrid- Red
CO- Blue
For a given amount of mass at high density, the CO produces more Iron group elements then the Hybrid model.  
The rectangular regions show one stanadard deviation from the average for both the C/O and hybrid 
models (SEE DON'S PAPER)
}
\end{figure}

\begin{figure}
\includegraphics[width=\columnwidth]{figures/compare.png}
\caption{\label{fig:compare}
The conversion of Ni56 to Iron group elements are more pronounced in my case. There are two assumptions as to why figure 6 is more pronounced in my paper then in Dons. 
First is that higher critical density leads to higher electron capture resulting in less Ni56 produced. Figure 4 shows that CO model has a stronger electron capture then the hybrid model. The second is that the central ignition leads to early burning at higher density. Both of these cases lead to more electron capture resulting in a larger difference in evolution between the models.
NEW: Production of 56Ni and mass burned to IGEs for CO (red) and
hybrid CONe (green) WD realizations. SAME AS DON'S FIG 18
}
\end{figure}

\begin{figure}
\includegraphics[width=\columnwidth]{figures/compare_burned_mass.png}
\caption{\label{fig:compare}
SAME AS DON'S FIGURE 19.
C/O blue hybrid red.
Note that one of the C/O simulations presented did not complete.
}
\end{figure}

\begin{figure}
\includegraphics[width=\columnwidth]{figures/compare_ratio.png}
\caption{\label{fig:compare_ratio}
Figure 8 shows the evolution of the materials for both models. It is determined that the CO model had the greatest amount of Ni56, then the other IGE materials, compared to the hybrid model.  
SAME AS DON'S FIGURE 20.
}
\end{figure}

\section{Discussion and Conclusion}


The conversion of \Ni{56} to Iron group elements are more pronounced in my
case. There are two assumptions as to why figure 6 is more pronounced
in my paper then in Dons.  First is that higher critical density leads
to higher electron capture resulting in less \Ni{56} produced. The second
is that the central ignition leads to early burning at higher density.


As the star burns mass, the surface area grows, resulting in a decrease
in density. The results show that the models that burned less mass during
the deflagration phase, expand less, produce more \Ni{56}. This is because
the density rich mass the model has at the DDT point is the ‘fuel’
it needs to produce \Ni{56}. Therefore, the more mass, the more Ni-56. On
average, the CO model, shown in figure 4, reaches the DDT faster then
the hybrid model. Figures 3 and 4 show that the hybrid model has a wider
range of estimated \Ni{56} mass suggesting a greater range of burned mass
and expansion during the deflagration phase.

This work was supported in part by the Department of Energy under
grant DE-FG02-87ER40317. The software used in this work was in part
developed by the DOE-supported ASC/Alliances Center for Astrophysical
Thermonuclear Flashes at the University of Chicago. Results in this
paper were obtained using the high-performance computing system at the
Institute for Advanced Computational Science at Stony Brook
University.  other grants...

%%%%%%%%%%%%%%%%%%%%%%%%%%%%%%%%%%%%%%%%%%%%%%%%%%%%%%%%%%%%%%%%%%
% SOFTWARE
%% \software{
%%   FLASH \citep{Fryxetal00},
%%   CASTRO \citep{castro1},
%%   MESA \citep{mesa1},
%%   Matplotlib \citep{http://dx.doi.org/10.5281/zenodo.44579}
%% }

%%%%%%%%%%%%%%%%%%%%%%%%%%%%%%%%%%%%%%%%%%%%%%%%%%%%%%%%%%%%%%%%%%
\bibliography{master}


\end{document}

